\subsection{Décisions sur l'implémentation}
%INTRODUCTION

Il aura bien fallu trois réunions pour ce mettre d'accord sur la structure que vas adopter le Générateur Pseudo Aléatoire de Scénario d'Aventure.


La  problématique était d'arriver à une structure facilitant la partie aléatoire de l'implémentation, tout en permettant de garder un jeu cohérent sous tout point de vue.
Qui dit aléatoire, dit arbre binaire ou divers choix possible. Donc la structure doit pouvoir adopter la possibilité d'offrir plusieurs chemins, plusieurs choix.


Solution trouvé: une structure composé de sous structures.


A première vu, cette solution parait suffisante et efficace. la décomposition en plusieurs sous-structures permet de briser un de nos principaux problème, la linéarité du jeu. Il faudrait imaginer les sous-structures comme des chemins différents possibles pour le joueur. La partie implémentation aléatoire du programme se charge d'établir la route entre ces divers sous-structures, nous ne rentrerons pas dans les détails ici.


Très vite, nous nous sommes rendu compte que décomposer le squelette du jeu en sous-structure ne suffisait pas pour arriver à donner l'illusion d'un jeu aléatoire au joueur. les possibilités offertes par les sous-structures étaient très intéressante mais pas assez.


Ce qui nous a paru vite évident, se reposant sur le même principe, c'est le besoin de décomposer les structures en sous-structures qui eux même sont décomposer en sous-structures. Plus le degré de profondeur des sous-ensembles est important, plus les chemins sont variés et l'illusion de l'aléatoire présente. Mais aussi plus c'est profond, plus l'implémentation devient complexe et périlleux.


Nous avons estimé le juste équilibre entre complexité et facilité pour l'aléatoire ainsi:


La structure générale est composé d'Univers. Ces Univers sont eux-même constitué de Mondes. Ces derniers sont eux aussi constitué de zone et enfin chaque zone a plusieurs questions.


Il existe une Zone de départ dans un Monde de départ dans un Univers.
Nous avons choisis ce système de départ pour obliger le joueur à se trouver dans une zone précise, dans le but de former le joueur au jeu. C'est en quelque sorte le tutoriel du début du jeu.


C'est bien la seule partie du jeu qui ne sera pas aléatoire.