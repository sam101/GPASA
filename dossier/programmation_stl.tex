\subsubsection{L'utilisation de la bibliothèque standard C++: la STL}
Nous avons choisi pour tout le coeur du programme (c'est à dire tout le programme excepté l'interface graphique) d'utiliser la bibliothèque standard du C++, la STL (Standard template library).

Bien sur, nous avons utilisé std::string  dans le cadre du stockage des chaines de caractères, mais le projet nous a aussi permis d'utiliser les conteneurs standard de la STL, tels std::vector ou std::map, qui sont plus haut niveau que de simples tableaux.

Nous avons choisi d'utiliser cette bibliothèque car celle-ci est disponible dans tout les compilateurs C++ à peut près récents, elle est portable et fonctionne donc sur n'importe quelle plateforme: ce qui nous affranchit donc de la dépendance à une bibliothèque qui nous empêcherait de faire un portage vers une autre plateforme.