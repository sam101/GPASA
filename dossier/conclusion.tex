\section{Conclusion}



\subsection{Ce qui aurait pu etre amélioré}

Par manque de temps, nous n'avons pu réaliser certaine amelioration du jeu qui auraient permis de le rendre plus complet. Ces améliorations sont les suivantes: 
\begin{itemize}
	\item Les points de vie du héros n'ont pu être implémentés. Actuellement,selon ses choix, le joueur est soit envoyé vers une nouvelle question, soit vers un écran de fin de jeu. Les points de vie auraient permis d'assouplir le gameplay.
	\item Certains attributs comme le nom, le prénom et l'âge du héros, n'ont pas été suffisamment inclus dans le jeu. Le joueur peut bien sûr les renseigner mais ils n'auront pas d'incidences au cours de la partie. À l'origine, nous avions par exemple prévu d'implémenter différentes tranches d'âges ce qui aurait permis de mieux adapter les situations au personnage.
	\item la mise en place d'une sauvegarde qui aurait permis au joueur de reprendre la partie là ou il se serait arrêter.
	\item L'ajout de monstres auquel le joueur serait confronté et en fonction de ses actions, il y aurait une baisse de ses point de vie ou celui du monstre.
	\item Nous aurions pu rajouter un inventaire que le joueur aurait pu consulter en cours de route
	\item La mise en place d'une carte du monde qui indiquerait au joueur où celui-ci se trouve dans le monde.
	\item Le raffinement de l'interface graphique, qui est relativement simple et qui est par exemple relativement peu adaptée aux appareils mobiles, où qui rend assez mal sous Mac OS X.
\end{itemize}
  

 faute de temps, 


\subsection{Conclusion générale}

Ce projet nous a aidé à approfondir les notions liées à la programmation orienté objet en C++.

Par ailleurs, l'utilisation du framework QT a permis une nouvelle approche en matière de création d'interface graphique. L'utilisation de subversion nous a également permis de nous initier aux systèmes de gestion de version.
 
L'importance de ce projet par rapport aux exercices habituels nous a incité à être efficace au possible. Dans cette optique nous avons découvert et utilisé la syntaxte Doxygen pour nos commentaires afin d'obtenir rapidement une documentation complète.
